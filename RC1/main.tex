\documentclass{beamer}
\usepackage[utf8]{inputenc}
\usepackage{xeCJK} % 引入之前安装的 xecjk 包
\usepackage[T1]{fontenc}
\usepackage{mathabx}
\usepackage{amsmath} 
\usepackage{mathpazo}
%\usepackage{eulervm}
%\usepackage{natbib}
\usepackage{bibentry}

%\setCJKmainfont{SourceHanSansCN-Light}
%\usetheme{Pittsburgh}
\usetheme{Boadilla}
%\usefonttheme{professionalfonts}
\usecolortheme{wolverine}
\useoutertheme{miniframes}
%\setbeamertemplate[miniframes theme]{headline}
% \usepackage[backend=bibtex,sorting=none]{biblatex}
% \addbibresource{example.bib} %BibTeX数据文件及位置
% \setbeamerfont{footnote}{size=\tiny}


\title{VP160 Recitation Class I}
% \subtitle{}
\author{Zeyi Ren}
\institute{UM-SJTU Joint Institute}

\begin{document}

\maketitle

\frame{\tableofcontents}

\begin{frame}{Before We Start}
  \begin{block}{What's in RC}
    \begin{enumerate}
      \item Basically two parts, conceptual part and exercises.
      \item A brief review of concepts.
      \item Exercises problems with hand-written notes, including some useful models that you may use in your assignments and exams.
    \end{enumerate}
  \end{block}
  \begin{block}{What you can do}
    \begin{enumerate}
      \item Stop me and ask question directly at any time.
      \item Basically anything but do not disturb your classmates.
      \item Come to my OH just after RC, if you need one-to-one help.
    \end{enumerate}
  \end{block}
  \textbf{The fewer fundamental principles you use to solve problems, the better your understanding of physics.}
    % \begin{exampleblock}{Example}<only@2->
    %   The set $\{1,2,3,5\}$ has four elements.
    %   \end{exampleblock}
    %   \begin{alertblock}{Wrong Theorem}
    %     $1=2$.
    %     \end{alertblock}
\end{frame}

\section{Units}
\begin{frame}
  \begin{enumerate}[1.]
    \item Scientific Notations
    \begin{itemize}
      \item $ a\times 10^n$ ($1\leq |a| < 10$)
      \item Usually used for very large number or very small number like computation in Astrophysics or Quantum Mechanics.
      % \item e.g.\\Gravitation constant $G = 6.67384\times 10^{-11} N\cdot m^2\cdot kg^{-2}$\\ Planck constant $h = 6.62606896\times 10^{-34} J\cdot s$
    \end{itemize}
    \item Unit Prefix and Conversion
    \begin{itemize}
      \item \textcolor{red}{$k$} (unit prefix) \textcolor{blue}{$m$} (unit)
      \item Some commonly-used unit prefixes:\\ 
      \begin{table}[H]
        \begin{tabular}{c|c|c|c|c|c|c|c}
        \hline
        $p$        & $n$       & $\mu$     & $m$       & $c$       & $k$      & $M$      & $G$      \\ \hline
        $10^{-12}$ & $10^{-9}$ & $10^{-6}$ & $10^{-3}$ & $10^{-2}$ & $10^{3}$ & $10^{6}$ & $10^{9}$ \\ \hline
        \end{tabular}
        \end{table}
        
        CASIO fx-991CN X $\rhd$ $\boxed{OPTN}$ $\rhd$ $3:$工程符号\\
        e.g.\\Input: 1m; Output: $\frac{1}{1000}$
    \end{itemize}
    \item Basic Units \& Derived Units
    \begin{itemize}
      \item SI system of units: 
      \begin{table}[H]
        \begin{tabular}{c|c|c|c|c|c|c|c}
        \hline
        Quantity & $L$        & $m$       & $t$       & $I$       & $T$       & $n$      & $lv$     \\ \hline
        Unit     & \textcolor{black}{$m$} & $kg$& $s$ & $A$ & $K$ & $mol$ & $cd$ \\ \hline
        \end{tabular}
        \end{table}
    \end{itemize}
  \end{enumerate}
\end{frame}

\begin{frame}{Dimensional Analysis: System of Units}
  % \begin{block}{Dimensional Analysis: System of Units}
\begin{enumerate}
  \item We can first select some physical quantities as the ``\textbf{basic quantities}”
  and specify a ``\textbf{unit for measurement}” for each basic quantity,
  the other physical quantities' units can be derived from the
  relation between them and the fundamental
  quantities. These physical quantities are called \textbf{derived quantities}
  and their units It's called \textbf{derived unit}.
  \item A set of units derived in this way, is called a \textbf{system of units}.
  \item We often use capital letter to represent a ``dimensional quantity", and
  use [$x$] to represent the ``dimensional quantity" of specific physical quantity $x$.\\
  e.g. The dimensional quantity of a particle of mass $m$ is written as: $M$ = [$m$].
\end{enumerate}
% \end{block}
\end{frame}

\begin{frame}
  \textcolor{blue}{Exercise 1}\\
  A simple pendulum consists of a light inextensible string AB with length $l$, with the end $A$ fixed, and a point mass $m$ attached to $B$. The pendulum oscillates with a small amplitude, and the period of oscillation is $T$. It is suggested that $T$ is proportional to the product of powers of $m$, $l$ and $g$, where $g$ is the acceleration due to gravity. Use dimensional analysis to find this relationship. 
\end{frame}

\section{Uncertainty and Significant Figures}
\begin{frame}{Uncertainty}
% \begin{block}{Uncertainty}
  \begin{enumerate}
    \item Because of limitations of measurement devices, imperfect
    measurement procedures and randomness of environmental
    conditions, as well as human factors related to the experimenter
    himself, no measurement can ever be perfect. Its result may therefore
    only be treated as an estimate of what we call the ``exact value” of a
    physical quantity. The experiment may both overestimate and
    underestimate the value of the physical quantity, and it is crucial to
    provide a measure of the error, or better uncertainty, that a result of
    the experiment carries (cited from ``Introduction to Measurement
    Data Analysis” in VP141).
    \item The detailed calculation will be encountered in VP141. The principles of uncertainty analysis will be explained in VE401.
  \end{enumerate}
% \end{block}
\end{frame}

\begin{frame}{Significant Figures} 
  % \begin{block}{Significant Figures}
    \begin{enumerate}
      \item Experimental uncertainty should almost always be rounded to one significant figure. The only exception is when the uncertainty has a leading digit of 1, then we can keep two significant figures (you may encounter this in VP141).
      \item The significant figure required in VP160 is not so strict.
    \end{enumerate} 
    \textcolor{blue}{Example}
    \begin{enumerate}
      \item $1.7392$ ($SF = 5$)
      \item $0.0970$ ($SF = 3$)
      \item $3.7\times 10^5$ ($SF = 2$)
      \item $5.00\times 10^3$ ($SF = 3$)
    \end{enumerate}
  % \end{block}
\end{frame}

\section{Fermi Problems}
\begin{frame}{Back-of-the-envelop Problems}
  \begin{block}{Definition}
    A quick estimation of some physical quantities. Also called ``Fermi problems''. Named after the American physicist Enrico Fermi.
  \end{block}
  \begin{block}{Tips}
    \begin{enumerate}
      \item Try to remember the order of magnitude of some important constant.
      \item This type of questions may occur in exams.
    \end{enumerate}
  \end{block}
  \textcolor{blue}{Exercise 2.1}\\
  True or False: The power of China Railway High-speed (CRH) is about 500kW.\\
  (False)\\
\end{frame}

\begin{frame}
  \textcolor{blue}{Exercise 2.2} (just for fun)\\
  Fermi's estimation of nuclear explosion: Fermi held a scrap of paper at the height of his head and released while at the same time, an atomic bomb exploded in New Mexico, the scrap of paper fell on the floor with a horizontal displacement of $2.5m$. How many tons of TNT this explosion is equivalent to?
\end{frame}

\section{Vectors}
\begin{frame}
  \begin{block}{Difinition}
    Vectors are quantities that have both \textcolor{red}{magnitude} and \textcolor{red}{direction}.
  \end{block}
  \begin{table}[]
    \begin{tabular}{c|c}
    \hline
    \textbf {Scalar}                                                      & \textbf {Vector}                                                             \\ \hline
    \begin{tabular}[c]{@{}c@{}}Distance\\ Speed\end{tabular} & \begin{tabular}[c]{@{}c@{}}Displacement\\ Velocity\end{tabular} \\ \hline
    \end{tabular}
    \end{table}
    \begin{block}{Vectors in $\mathbb{R} ^n$}
      $\vec{u} = $ ($u_1,u_2,...,u_n $)$^T$
    \end{block}
\end{frame}

\begin{frame}{Basic Vector operations}
  \begin{itemize}
    \item Addition \& Subtraction\\ $$\vec{u}\pm \vec{v}=(u_1\pm v_1,u_2\pm v2,u_3\pm v_3)$$
    \item Scalar Multiplication$$\lambda \vec{u} = (\lambda u_1, \lambda u_2, \lambda u_3)$$
    \item Dot Product$$\vec{u}\cdot \vec{v} = |u||v|cos\theta=u_1v_1+u_2v_2+u_3v_3$$
    \item Orthogonal Projection Vector of the vector $\vec{u}$ onto the vector $\vec{v}$$$\frac{\vec{u}\cdot \vec{v}}{|\vec{v}|}\cdot \frac{\vec{v}}{|\vec{v}|}$$
  \end{itemize}
\end{frame}

\begin{frame}{Basic Vector operations}
  \begin{itemize}
    \item Cross Product
    \begin{itemize}
      \item Magnitude: $|\vec{u}\times \vec{v}|=|\vec{u}||\vec{v}|sin\theta$
      \item Direction: determined by \textbf{Right Hand Rule}
      \item Matrix expression(Using determinant):
      \begin{align*}
        \vec{u}\times \vec{v} &= \left|\begin{matrix}\hat{i}&\hat{j}&\hat{k}\\u_1 & u_2 & u_3\\ v_1 & v_2 & v_3 \end{matrix}\right|\\&=(u_2v_3-u_3v_2)\hat{i}+(u_3v_1-u_1v_3)\hat{j}+ (u_1v_2-u_2v_1)\hat{k}
 \end{align*}
    \end{itemize}
    \item Triple Product
    \begin{itemize}
      \item Scalar Triple Product:$$\vec{u}\cdot(\vec{v}\times \vec{w})=\vec{v}\cdot(\vec{w}\times \vec{u})=\vec{w}\cdot(\vec{u}\times \vec{v})$$
      \item Vector Triple Product:$$\vec{u}\times (\vec{v}\times \vec{w})=\vec{v}(\vec{u}\cdot\vec{w})-\vec{w}(\vec{u}\cdot \vec{v})$$
    \end{itemize}
  \end{itemize}
\end{frame}

\begin{frame}
\textcolor{blue}{Exercise 3.1}

Consider two vectors $ \vec{u}= 3\hat{n_x} +4\hat{n_y}$, and $\vec{w}= 6\hat{n_x} +16\hat{n_y}$. Find:\\
(a) the components of the vector $\vec w$ that are parallel and perpendicular
to the vector $\vec u$,\\
(b) the angle between $\vec w$ and $\vec u$.
\end{frame}

\begin{frame}
\textcolor{blue}{Exercise 3.2}

Check that in Cartesian coordinates, the two expression equations
of dot product of two vectors $\vec{u}$ and $\vec{v}$ are equivalent:$$\vec{u}\cdot \vec{v} = u_1v_1 + u_2v_2 + u_3v_3$$
$$\vec{u}\cdot \vec{v} = |\vec{u}||\vec{v}|cos\theta$$
\end{frame}

\begin{frame}
\textcolor{blue}{Exercise 3.3}

Prove:$$
\vec{u}\times (\vec{v}\times \vec{w})+\vec{v}\times(\vec{w}\times \vec{u})+\vec{w}\times(\vec{u}\times\vec{v})=0$$
Then check \textbf{under what circumstance} the associativity of vector triple product stands, that is $$\vec{u}\times (\vec{v}\times \vec{w}) = (\vec{u}\times \vec{v})\times \vec{w}$$
(In general, this equation does not hold)
\end{frame}

\section{3D Curvilinear Coordinate system}
\begin{frame}{Cartesian Coordinate}
  \begin{enumerate}
    \item Coordinates: $x, y, z$
    \item Unit Vectors: $\hat{n_x}, \hat{n_y}, \hat{n_z}$
    \item Position Vector: $\vec{r} = x\hat{n_x}+y\hat{n_y}+z\hat{n_z}$
  \end{enumerate}
\end{frame}

\begin{frame}{Cylindrical Coordinate}
  \begin{enumerate}
    \item Coordinates: $\rho, \phi, z$
    \item Unit Vectors: $\hat{n_\rho}, \hat{n_\phi}, \hat{n_z}$
    \item Position Vectors: $\vec{r} = \rho\hat{n_\rho}+z\hat{n_z}$
    \item Relationship with Cartesian Coordinates:
    $$\left\{\begin{array} { l } 
      { \rho = \sqrt { x ^ { 2 } + y ^ { 2 } } } \\
      { \phi = \operatorname { a r c t a n } ( y / x ) } \\
      { z = z }
      \end{array} \quad \left\{\begin{array}{l}
      x=\rho \cos (\theta) \\
      y=\rho \sin (\phi) \\
      z=z
      \end{array}\right.\right.$$
  \end{enumerate}
\end{frame}

\begin{frame}{Spherical Coordinate}
  \begin{enumerate}
    \item Coordinates: $r, \phi, \theta$
    \item Unit Vectors: $\hat{n_r}, \hat{n_\phi}, \hat{n_{\theta}}$
    \item Position Vector: $\vec{r} = r\hat{n_r}$
    \item Relationship with Cartesian Coordinate:$$\left\{\begin{array} { l } 
      { r = \sqrt { x ^ { 2 } + y ^ { 2 } + z ^ { 2 } } } \\
      { \phi = \operatorname { a r c t a n } ( y / x ) } \\
      { \theta = \operatorname { a r c t a n } ( \sqrt { x ^ { 2 } + y ^ { 2 } } / z ) }
      \end{array} \quad \left\{\begin{array}{l}
      x=\rho \sin (\theta) \cos (\phi) \\
      y=r \sin (\theta) \sin (\phi) \\
      z=r \cos (\theta)
      \end{array}\right.\right.$$
  \end{enumerate}
\end{frame}

\begin{frame}
\textcolor{blue}{Exercise 4.1}

Derive the above relation equations.
\end{frame}

\begin{frame}
\textcolor{blue}{Exercise 4.2}

Using Spherical Coordinate, calculate the volume of the solid-line part showed below
\begin{figure}[htbp]
\centering
\includegraphics[width=0.5 \linewidth, angle =0]{qiuque.png}
% \caption{Exercise 5.2}
% \label{fig:1}
\end{figure}
\end{frame}

\section{1D Kinematics}
\begin{frame}
  \begin{itemize}
    \item Average vs. Instantaneous Quantities$$v_{x,A} = \frac{x(t+\Delta t)-x(t)}{\Delta t}\quad a_{x,A} = \frac{v(t+\Delta t)-v(t)}{\Delta t}$$
    \item Relationships\begin{align*}
    x &= x(t)\\
    v &= v(t)=\frac{d}{dt} x(t)\\
    a &= a(t)=\frac{d}{dt} v(t)=\frac{d^2}{dt^2}x(t)  
    \end{align*}
    \item Relative motion$$x = x_r + x';\quad v = v_r+v';\quad a=a_r+a'$$
  \end{itemize}
\end{frame}

\begin{frame}
\textcolor{blue}{Exercise 5.1}

A car is moving in one direction along a straight line. Find the average velocity of the car if:\\
(a) it travels half of the journey time with velocity $v_1$ and the other
half with velocity $v_2$,\\
(b) it covers half of the distance with velocity $v_1$ and the other
one with velocity $v_2$.
\end{frame}

\begin{frame}
\textcolor{blue}{Exercise 5.2}

A particle is moving along a straight line with velocity $v_x(t) =
-\beta A\omega e^{-\beta}cos\omega t$, where $A, \omega$, and $\beta$ are positive constants. Assuming
that $x(0) = 5m$.\\
(a) What are the units of these constants?\\
(b) Find acceleration $a_x (t)$ and position $x(t)$ of the particle.\\
(c) Sketch the graphs of $x(t)$, $v_x (t)$, and $a_x (t)$.\\
(d) What kind of motion may these results describe?
\end{frame}

\begin{frame}
\textcolor{blue}{Exercise 5.3}

In a certain motion of a particle along a straight line, the acceleration
turns out to be related to the position of the particle according to the formula $a_x = \sqrt{kx}$, where $k > 0$ is a constant
and $x > 0$. How does the velocity depend on x, if we know that
for $v_x (x_0) = v_0$?
\end{frame}

\begin{frame}
\textcolor{blue}{Exercise 5.4}

A dripping water faucet steadily releases drops $1$drop/s apart. As
these drops fall, does the distance between them decrease, increase,
or remain the same? (Try to find the answer without calculation)
\end{frame}

\begin{frame}
\textcolor{blue}{Exercise 5.5}

A spare paddle drops from a fisherman’s canoe. After one hour
of paddling the fisherman realizes that the paddle is missing.
He turn around and paddles his canoe back to find the paddle.
Assume that the fisherman paddles always with the same speed
$v = 10km/h$ with respect to the river, the speed of the rivers
current is $u = 6km/h$. Find:\\
(a) the time that the fisherman takes to find the paddle;\\
(b) the distance between the places where the paddle drops and
the fisherman find it.\\
Can you find the answers within a second?
\end{frame}

\begin{frame}{Reference}
  \begin{thebibliography}{9}
  % \setbeamertemplate{bibliography item}[online]
  % \bibitem{A} ItemA
  % \setbeamertemplate{bibliography item}[book]
  % \bibitem{B} ItemB
  \setbeamertemplate{bibliography item}[article]
  \bibitem{C} Yigao Fang.\\
  \textcolor{black}{VP160 Recitation Slides.}\\
  2020
  \bibitem{C} Haoyang Zhang.\\
  \textcolor{black}{VP160 Recitation Slides.}\\
  2020
  % \setbeamertemplate{bibliography item}[book]
  % \bibitem{B} R. P. Feynman\\
  % \textit{\textcolor{black}{The Feynman Lectures on Physics}}
  % \setbeamertemplate{bibliography item}[triangle]
  % \bibitem{D} ItemD
  % \setbeamertemplate{bibliography item}[text]
  % \bibitem{E} ItemE
  \end{thebibliography}
  \end{frame}

  \end{document}



